%% Generated by Sphinx.
\def\sphinxdocclass{report}
\documentclass[letterpaper,10pt,english]{sphinxmanual}
\ifdefined\pdfpxdimen
   \let\sphinxpxdimen\pdfpxdimen\else\newdimen\sphinxpxdimen
\fi \sphinxpxdimen=.75bp\relax
\ifdefined\pdfimageresolution
    \pdfimageresolution= \numexpr \dimexpr1in\relax/\sphinxpxdimen\relax
\fi
%% let collapsible pdf bookmarks panel have high depth per default
\PassOptionsToPackage{bookmarksdepth=5}{hyperref}

\PassOptionsToPackage{booktabs}{sphinx}
\PassOptionsToPackage{colorrows}{sphinx}

\PassOptionsToPackage{warn}{textcomp}
\usepackage[utf8]{inputenc}
\ifdefined\DeclareUnicodeCharacter
% support both utf8 and utf8x syntaxes
  \ifdefined\DeclareUnicodeCharacterAsOptional
    \def\sphinxDUC#1{\DeclareUnicodeCharacter{"#1}}
  \else
    \let\sphinxDUC\DeclareUnicodeCharacter
  \fi
  \sphinxDUC{00A0}{\nobreakspace}
  \sphinxDUC{2500}{\sphinxunichar{2500}}
  \sphinxDUC{2502}{\sphinxunichar{2502}}
  \sphinxDUC{2514}{\sphinxunichar{2514}}
  \sphinxDUC{251C}{\sphinxunichar{251C}}
  \sphinxDUC{2572}{\textbackslash}
\fi
\usepackage{cmap}
\usepackage[T1]{fontenc}
\usepackage{amsmath,amssymb,amstext}
\usepackage{babel}



\usepackage{tgtermes}
\usepackage{tgheros}
\renewcommand{\ttdefault}{txtt}



\usepackage[Bjarne]{fncychap}
\usepackage{sphinx}

\fvset{fontsize=auto}
\usepackage{geometry}


% Include hyperref last.
\usepackage{hyperref}
% Fix anchor placement for figures with captions.
\usepackage{hypcap}% it must be loaded after hyperref.
% Set up styles of URL: it should be placed after hyperref.
\urlstyle{same}

\addto\captionsenglish{\renewcommand{\contentsname}{Installation}}

\usepackage{sphinxmessages}
\setcounter{tocdepth}{1}



\title{FC\sphinxhyphen{}DFT}
\date{Jun 12, 2025}
\release{1.0.0}
\author{Jun\sphinxhyphen{}Hyeong Kim}
\newcommand{\sphinxlogo}{\vbox{}}
\renewcommand{\releasename}{Release}
\makeindex
\begin{document}

\ifdefined\shorthandoff
  \ifnum\catcode`\=\string=\active\shorthandoff{=}\fi
  \ifnum\catcode`\"=\active\shorthandoff{"}\fi
\fi

\pagestyle{empty}
\sphinxmaketitle
\pagestyle{plain}
\sphinxtableofcontents
\pagestyle{normal}
\phantomsection\label{\detokenize{index::doc}}


\sphinxstepscope


\chapter{Installation}
\label{\detokenize{install:installation}}\label{\detokenize{install::doc}}
\sphinxAtStartPar
This page explains how to install FC\sphinxhyphen{}DFT code.


\section{Requirements}
\label{\detokenize{install:requirements}}\begin{itemize}
\item {} 
\sphinxAtStartPar
GeomeTRIC

\item {} 
\sphinxAtStartPar
PySCF

\item {} 
\sphinxAtStartPar
PyAMG

\item {} 
\sphinxAtStartPar
PyAMGCL

\item {} 
\sphinxAtStartPar
GPU4PySCF (optional)

\end{itemize}


\section{How to install}
\label{\detokenize{install:how-to-install}}
\sphinxAtStartPar
Download the lastest version of FC\sphinxhyphen{}DFT from the repository:
\begin{enumerate}
\sphinxsetlistlabels{\arabic}{enumi}{enumii}{}{.}%
\item {} 
\sphinxAtStartPar
Clone the repository:

\begin{sphinxVerbatim}[commandchars=\\\{\}]
\PYGZdl{} git clone https://github.com/Yang\PYGZhy{}Laboratory/FC\PYGZhy{}DFT.git
\end{sphinxVerbatim}

\item {} 
\sphinxAtStartPar
Change the directory to the cloned repository:

\begin{sphinxVerbatim}[commandchars=\\\{\}]
\PYGZdl{} cd FC\PYGZhy{}DFT
\end{sphinxVerbatim}

\item {} 
\sphinxAtStartPar
Install the package using \sphinxtitleref{pip}:

\begin{sphinxVerbatim}[commandchars=\\\{\}]
\PYGZdl{} pip install .
\end{sphinxVerbatim}

\item {} 
\sphinxAtStartPar
Change directory to \sphinxtitleref{\$PYTHONPATH/fcdft/lib} and create build directory.

\item {} 
\sphinxAtStartPar
Go into \sphinxtitleref{build} and compile the C shared libraries by \sphinxtitleref{cmake ..} and \sphinxtitleref{make}.

\end{enumerate}

\sphinxstepscope


\chapter{Examples}
\label{\detokenize{examples:examples}}\label{\detokenize{examples::doc}}
\sphinxAtStartPar
This chapter introduces brief exmaples of running FC\sphinxhyphen{}DFT and Poisson\sphinxhyphen{}Boltzmann solvation calculations


\section{Wide\sphinxhyphen{}Band Limit}
\label{\detokenize{examples:wide-band-limit}}
\sphinxAtStartPar
FC\sphinxhyphen{}DFT uses the wide\sphinxhyphen{}band limit (WBL) approximation. In particular, the WBL\sphinxhyphen{}Molecule approximation is adopted for simple but powerful implementation. WBL\sphinxhyphen{}Molecule requires users to provide the imaginary part of the self\sphinxhyphen{}energy in \sphinxcode{\sphinxupquote{WBLMolecule}} object.
Currently, spin\sphinxhyphen{}restricted version of FC\sphinxhyphen{}DFT (\sphinxcode{\sphinxupquote{fcdft.wbl.rks}}) is supported. Below is a sample code, where the self\sphinxhyphen{}energy of 0.01 eV is attached to the sulfur atom of methanethiol with 25.95 electrons.

\begin{sphinxVerbatim}[commandchars=\\\{\}]
\PYG{g+gp}{\PYGZgt{}\PYGZgt{}\PYGZgt{} }\PYG{k+kn}{from}\PYG{+w}{ }\PYG{n+nn}{pyscf}\PYG{+w}{ }\PYG{k+kn}{import} \PYG{n}{gto}
\PYG{g+gp}{\PYGZgt{}\PYGZgt{}\PYGZgt{} }\PYG{k+kn}{from}\PYG{+w}{ }\PYG{n+nn}{pyscf}\PYG{n+nn}{.}\PYG{n+nn}{dft}\PYG{+w}{ }\PYG{k+kn}{import} \PYG{n}{RKS}
\PYG{g+gp}{\PYGZgt{}\PYGZgt{}\PYGZgt{} }\PYG{n}{mol} \PYG{o}{=} \PYG{n}{gto}\PYG{o}{.}\PYG{n}{M}\PYG{p}{(}\PYG{n}{atom}\PYG{o}{=}\PYG{l+s+s1}{\PYGZsq{}\PYGZsq{}\PYGZsq{}}
\PYG{g+gp}{... }\PYG{l+s+s1}{           C       \PYGZhy{}1.718553971     \PYGZhy{}0.000000250     \PYGZhy{}0.626147715}
\PYG{g+gp}{... }\PYG{l+s+s1}{           H       \PYGZhy{}2.739245971     \PYGZhy{}0.008907250     \PYGZhy{}0.227127715}
\PYG{g+gp}{... }\PYG{l+s+s1}{           H       \PYGZhy{}1.200493971     \PYGZhy{}0.879491250     \PYGZhy{}0.227127715}
\PYG{g+gp}{... }\PYG{l+s+s1}{           H       \PYGZhy{}1.215921971      0.888398750     \PYGZhy{}0.227127715}
\PYG{g+gp}{... }\PYG{l+s+s1}{           S       \PYGZhy{}1.718553971     \PYGZhy{}0.000000250     \PYGZhy{}2.396147715}
\PYG{g+gp}{... }\PYG{l+s+s1}{           H       \PYGZhy{}2.150082583      0.805150681     \PYGZhy{}2.710667448}\PYG{l+s+s1}{\PYGZsq{}\PYGZsq{}\PYGZsq{}}\PYG{p}{,}
\PYG{g+gp}{... }       \PYG{n}{charge}\PYG{o}{=}\PYG{l+m+mi}{0}\PYG{p}{,} \PYG{n}{basis}\PYG{o}{=}\PYG{l+s+s1}{\PYGZsq{}}\PYG{l+s+s1}{6\PYGZhy{}31g**}\PYG{l+s+s1}{\PYGZsq{}}\PYG{p}{)}
\PYG{g+gp}{\PYGZgt{}\PYGZgt{}\PYGZgt{} }\PYG{n}{mf} \PYG{o}{=} \PYG{n}{RKS}\PYG{p}{(}\PYG{n}{mol}\PYG{p}{,} \PYG{n}{xc}\PYG{o}{=}\PYG{l+s+s1}{\PYGZsq{}}\PYG{l+s+s1}{pbe}\PYG{l+s+s1}{\PYGZsq{}}\PYG{p}{)}
\PYG{g+gp}{\PYGZgt{}\PYGZgt{}\PYGZgt{} }\PYG{n}{mf}\PYG{o}{.}\PYG{n}{kernel}\PYG{p}{(}\PYG{p}{)}
\PYG{g+gp}{\PYGZgt{}\PYGZgt{}\PYGZgt{} }\PYG{k+kn}{from}\PYG{+w}{ }\PYG{n+nn}{fcdft}\PYG{n+nn}{.}\PYG{n+nn}{wbl}\PYG{n+nn}{.}\PYG{n+nn}{rks}\PYG{+w}{ }\PYG{k+kn}{import} \PYG{o}{*}
\PYG{g+gp}{\PYGZgt{}\PYGZgt{}\PYGZgt{} }\PYG{n}{wblmf} \PYG{o}{=} \PYG{n}{WBLMolecule}\PYG{p}{(}\PYG{n}{mf}\PYG{p}{,} \PYG{n}{broad}\PYG{o}{=}\PYG{l+m+mf}{0.01}\PYG{p}{,} \PYG{n}{nelectron}\PYG{o}{=}\PYG{l+m+mf}{25.95}\PYG{p}{)}
\PYG{g+gp}{\PYGZgt{}\PYGZgt{}\PYGZgt{} }\PYG{n}{wblmf}\PYG{o}{.}\PYG{n}{kernel}\PYG{p}{(}\PYG{p}{)}
\end{sphinxVerbatim}


\section{Non\sphinxhyphen{}Linear Poisson\sphinxhyphen{}Boltzmann Solvation Model}
\label{\detokenize{examples:non-linear-poisson-boltzmann-solvation-model}}
\sphinxAtStartPar
We provide the Poisson\sphinxhyphen{}Boltzmann solver for general purpose. \sphinxcode{\sphinxupquote{fcdft.solvent.pbe}} module supports usual solvation energy calculations as what polarizable continuum model does.
To do so, a few attributes of \sphinxcode{\sphinxupquote{PBE}} needs to be controlled since it was originally intended to solve the electrostatic potential under the Gouy\sphinxhyphen{}Chapman\sphinxhyphen{}Stern theory:

\begin{sphinxVerbatim}[commandchars=\\\{\}]
\PYG{g+gp}{\PYGZgt{}\PYGZgt{}\PYGZgt{} }\PYG{k+kn}{from}\PYG{+w}{ }\PYG{n+nn}{pyscf}\PYG{+w}{ }\PYG{k+kn}{import} \PYG{n}{gto}
\PYG{g+gp}{\PYGZgt{}\PYGZgt{}\PYGZgt{} }\PYG{k+kn}{from}\PYG{+w}{ }\PYG{n+nn}{pyscf}\PYG{n+nn}{.}\PYG{n+nn}{dft}\PYG{+w}{ }\PYG{k+kn}{import} \PYG{n}{RKS}
\PYG{g+gp}{\PYGZgt{}\PYGZgt{}\PYGZgt{} }\PYG{n}{mol} \PYG{o}{=} \PYG{n}{gto}\PYG{o}{.}\PYG{n}{M}\PYG{p}{(}\PYG{n}{atom}\PYG{o}{=}\PYG{l+s+s1}{\PYGZsq{}\PYGZsq{}\PYGZsq{}}
\PYG{g+gp}{... }\PYG{l+s+s1}{           O        0.152427064      0.959723218     \PYGZhy{}2.275350162}
\PYG{g+gp}{... }\PYG{l+s+s1}{           H        0.152427064      1.719060218     \PYGZhy{}1.679307162}
\PYG{g+gp}{... }\PYG{l+s+s1}{           H        0.152427064      0.200386218     \PYGZhy{}1.679307162}\PYG{l+s+s1}{\PYGZsq{}\PYGZsq{}\PYGZsq{}}\PYG{p}{,}
\PYG{g+gp}{... }       \PYG{n}{charge}\PYG{o}{=}\PYG{l+m+mi}{0}\PYG{p}{,} \PYG{n}{basis}\PYG{o}{=}\PYG{l+s+s1}{\PYGZsq{}}\PYG{l+s+s1}{6\PYGZhy{}31g**}\PYG{l+s+s1}{\PYGZsq{}}\PYG{p}{)}
\PYG{g+gp}{\PYGZgt{}\PYGZgt{}\PYGZgt{} }\PYG{n}{mf} \PYG{o}{=} \PYG{n}{RKS}\PYG{p}{(}\PYG{n}{mol}\PYG{p}{,} \PYG{n}{xc}\PYG{o}{=}\PYG{l+s+s1}{\PYGZsq{}}\PYG{l+s+s1}{b3lyp}\PYG{l+s+s1}{\PYGZsq{}}\PYG{p}{)}
\PYG{g+gp}{\PYGZgt{}\PYGZgt{}\PYGZgt{} }\PYG{k+kn}{from}\PYG{+w}{ }\PYG{n+nn}{fcdft}\PYG{n+nn}{.}\PYG{n+nn}{solvent}\PYG{n+nn}{.}\PYG{n+nn}{pbe}\PYG{+w}{ }\PYG{k+kn}{import} \PYG{o}{*}
\PYG{g+gp}{\PYGZgt{}\PYGZgt{}\PYGZgt{} }\PYG{n}{cm} \PYG{o}{=} \PYG{n}{PBE}\PYG{p}{(}\PYG{n}{mol}\PYG{p}{,} \PYG{n}{cb}\PYG{o}{=}\PYG{l+m+mf}{1.0}\PYG{p}{,} \PYG{n}{length}\PYG{o}{=}\PYG{l+m+mi}{15}\PYG{p}{,} \PYG{n}{ngrids}\PYG{o}{=}\PYG{l+m+mi}{41}\PYG{p}{)}
\PYG{g+gp}{\PYGZgt{}\PYGZgt{}\PYGZgt{} }\PYG{n}{cm}\PYG{o}{.}\PYG{n}{eps} \PYG{o}{=} \PYG{l+m+mf}{78.3553}
\PYG{g+gp}{\PYGZgt{}\PYGZgt{}\PYGZgt{} }\PYG{n}{cm}\PYG{o}{.}\PYG{n}{atom\PYGZus{}bottom} \PYG{o}{=} \PYG{l+s+s1}{\PYGZsq{}}\PYG{l+s+s1}{center}\PYG{l+s+s1}{\PYGZsq{}}
\PYG{g+gp}{\PYGZgt{}\PYGZgt{}\PYGZgt{} }\PYG{n}{cm}\PYG{o}{.}\PYG{n}{nelectron} \PYG{o}{=} \PYG{n}{mol}\PYG{o}{.}\PYG{n}{nelectron}
\PYG{g+gp}{\PYGZgt{}\PYGZgt{}\PYGZgt{} }\PYG{n}{cm}\PYG{o}{.}\PYG{n}{bias} \PYG{o}{=} \PYG{l+m+mf}{0.0e0}
\PYG{g+gp}{\PYGZgt{}\PYGZgt{}\PYGZgt{} }\PYG{n}{cm}\PYG{o}{.}\PYG{n}{surf} \PYG{o}{=} \PYG{l+m+mf}{0.0e0}
\PYG{g+gp}{\PYGZgt{}\PYGZgt{}\PYGZgt{} }\PYG{n}{solmf} \PYG{o}{=} \PYG{n}{pbe\PYGZus{}for\PYGZus{}scf}\PYG{p}{(}\PYG{n}{mf}\PYG{p}{,} \PYG{n}{cm}\PYG{p}{)}
\PYG{g+gp}{\PYGZgt{}\PYGZgt{}\PYGZgt{} }\PYG{n}{solmf}\PYG{o}{.}\PYG{n}{kernel}\PYG{p}{(}\PYG{p}{)}
\end{sphinxVerbatim}

\sphinxAtStartPar
The following example introduces how to run the Poisson\sphinxhyphen{}Boltzmann solver under the Gouy\sphinxhyphen{}Chapman\sphinxhyphen{}Stern boundary values:

\begin{sphinxVerbatim}[commandchars=\\\{\}]
\PYG{g+gp}{\PYGZgt{}\PYGZgt{}\PYGZgt{} }\PYG{k+kn}{from}\PYG{+w}{ }\PYG{n+nn}{pyscf}\PYG{+w}{ }\PYG{k+kn}{import} \PYG{n}{gto}
\PYG{g+gp}{\PYGZgt{}\PYGZgt{}\PYGZgt{} }\PYG{k+kn}{from}\PYG{+w}{ }\PYG{n+nn}{pyscf}\PYG{n+nn}{.}\PYG{n+nn}{dft}\PYG{+w}{ }\PYG{k+kn}{import} \PYG{n}{RKS}
\PYG{g+gp}{\PYGZgt{}\PYGZgt{}\PYGZgt{} }\PYG{n}{mol} \PYG{o}{=} \PYG{n}{gto}\PYG{o}{.}\PYG{n}{M}\PYG{p}{(}\PYG{n}{atom}\PYG{o}{=}\PYG{l+s+s1}{\PYGZsq{}\PYGZsq{}\PYGZsq{}}
\PYG{g+gp}{... }\PYG{l+s+s1}{           C       \PYGZhy{}1.718553971     \PYGZhy{}0.000000250     \PYGZhy{}0.626147715}
\PYG{g+gp}{... }\PYG{l+s+s1}{           H       \PYGZhy{}2.739245971     \PYGZhy{}0.008907250     \PYGZhy{}0.227127715}
\PYG{g+gp}{... }\PYG{l+s+s1}{           H       \PYGZhy{}1.200493971     \PYGZhy{}0.879491250     \PYGZhy{}0.227127715}
\PYG{g+gp}{... }\PYG{l+s+s1}{           H       \PYGZhy{}1.215921971      0.888398750     \PYGZhy{}0.227127715}
\PYG{g+gp}{... }\PYG{l+s+s1}{           S       \PYGZhy{}1.718553971     \PYGZhy{}0.000000250     \PYGZhy{}2.396147715}
\PYG{g+gp}{... }\PYG{l+s+s1}{           H       \PYGZhy{}2.150082583      0.805150681     \PYGZhy{}2.710667448}\PYG{l+s+s1}{\PYGZsq{}\PYGZsq{}\PYGZsq{}}\PYG{p}{,}
\PYG{g+gp}{... }       \PYG{n}{charge}\PYG{o}{=}\PYG{l+m+mi}{0}\PYG{p}{,} \PYG{n}{basis}\PYG{o}{=}\PYG{l+s+s1}{\PYGZsq{}}\PYG{l+s+s1}{6\PYGZhy{}31g**}\PYG{l+s+s1}{\PYGZsq{}}\PYG{p}{)}
\PYG{g+gp}{\PYGZgt{}\PYGZgt{}\PYGZgt{} }\PYG{n}{mf} \PYG{o}{=} \PYG{n}{RKS}\PYG{p}{(}\PYG{n}{mol}\PYG{p}{,} \PYG{n}{xc}\PYG{o}{=}\PYG{l+s+s1}{\PYGZsq{}}\PYG{l+s+s1}{pbe}\PYG{l+s+s1}{\PYGZsq{}}\PYG{p}{)}
\PYG{g+gp}{\PYGZgt{}\PYGZgt{}\PYGZgt{} }\PYG{n}{mf}\PYG{o}{.}\PYG{n}{kernel}\PYG{p}{(}\PYG{p}{)}
\PYG{g+gp}{\PYGZgt{}\PYGZgt{}\PYGZgt{} }\PYG{k+kn}{from}\PYG{+w}{ }\PYG{n+nn}{fcdft}\PYG{n+nn}{.}\PYG{n+nn}{wbl}\PYG{n+nn}{.}\PYG{n+nn}{rks}\PYG{+w}{ }\PYG{k+kn}{import} \PYG{o}{*}
\PYG{g+gp}{\PYGZgt{}\PYGZgt{}\PYGZgt{} }\PYG{n}{wblmf} \PYG{o}{=} \PYG{n}{WBLMolecule}\PYG{p}{(}\PYG{n}{mf}\PYG{p}{,} \PYG{n}{broad}\PYG{o}{=}\PYG{l+m+mf}{0.01}\PYG{p}{,} \PYG{n}{nelectron}\PYG{o}{=}\PYG{l+m+mf}{25.95}\PYG{p}{)}
\PYG{g+gp}{\PYGZgt{}\PYGZgt{}\PYGZgt{} }\PYG{n}{wblmf}\PYG{o}{.}\PYG{n}{kernel}\PYG{p}{(}\PYG{p}{)}
\PYG{g+gp}{\PYGZgt{}\PYGZgt{}\PYGZgt{} }\PYG{n}{dm} \PYG{o}{=} \PYG{n}{wblmf}\PYG{o}{.}\PYG{n}{make\PYGZus{}rdm1}\PYG{p}{(}\PYG{p}{)}
\PYG{g+gp}{\PYGZgt{}\PYGZgt{}\PYGZgt{} }\PYG{k+kn}{from}\PYG{+w}{ }\PYG{n+nn}{fcdft}\PYG{n+nn}{.}\PYG{n+nn}{solvent}\PYG{n+nn}{.}\PYG{n+nn}{pbe}\PYG{+w}{ }\PYG{k+kn}{import} \PYG{o}{*}
\PYG{g+gp}{\PYGZgt{}\PYGZgt{}\PYGZgt{} }\PYG{n}{cm} \PYG{o}{=} \PYG{n}{PBE}\PYG{p}{(}\PYG{n}{mol}\PYG{p}{,} \PYG{n}{cb}\PYG{o}{=}\PYG{l+m+mf}{1.0}\PYG{p}{,} \PYG{n}{length}\PYG{o}{=}\PYG{l+m+mi}{15}\PYG{p}{,} \PYG{n}{ngrids}\PYG{o}{=}\PYG{l+m+mi}{41}\PYG{p}{,} \PYG{n}{stern\PYGZus{}sam}\PYG{o}{=}\PYG{l+m+mf}{3.0}\PYG{p}{)}
\PYG{g+gp}{\PYGZgt{}\PYGZgt{}\PYGZgt{} }\PYG{n}{cm}\PYG{o}{.}\PYG{n}{eps} \PYG{o}{=} \PYG{l+m+mf}{78.3553}
\PYG{g+gp}{\PYGZgt{}\PYGZgt{}\PYGZgt{} }\PYG{n}{cm}\PYG{o}{.}\PYG{n}{eps\PYGZus{}sam} \PYG{o}{=} \PYG{l+m+mf}{2.284}
\PYG{g+gp}{\PYGZgt{}\PYGZgt{}\PYGZgt{} }\PYG{n}{cm}\PYG{o}{.}\PYG{n}{\PYGZus{}dm} \PYG{o}{=} \PYG{n}{dm}
\PYG{g+gp}{\PYGZgt{}\PYGZgt{}\PYGZgt{} }\PYG{n}{solmf} \PYG{o}{=} \PYG{n}{pbe\PYGZus{}for\PYGZus{}scf}\PYG{p}{(}\PYG{n}{wblmf}\PYG{p}{,} \PYG{n}{cm}\PYG{p}{)}
\PYG{g+gp}{\PYGZgt{}\PYGZgt{}\PYGZgt{} }\PYG{n}{solmf}\PYG{o}{.}\PYG{n}{kernel}\PYG{p}{(}\PYG{p}{)}
\end{sphinxVerbatim}


\section{Geometry Optimization}
\label{\detokenize{examples:geometry-optimization}}
\sphinxAtStartPar
Our code supports analytic nuclear gradients of FC\sphinxhyphen{}DFT as well as the Poisson\sphinxhyphen{}Boltzmann solvation model. We have tested geometry optimization using GeomeTRIC, an external geometry optimizer implemented in PySCF:

\begin{sphinxVerbatim}[commandchars=\\\{\}]
\PYG{g+gp}{\PYGZgt{}\PYGZgt{}\PYGZgt{} }\PYG{k+kn}{from}\PYG{+w}{ }\PYG{n+nn}{pyscf}\PYG{n+nn}{.}\PYG{n+nn}{geomopt}\PYG{n+nn}{.}\PYG{n+nn}{geometric\PYGZus{}solver}\PYG{+w}{ }\PYG{k+kn}{import} \PYG{n}{optimize}
\PYG{g+gp}{\PYGZgt{}\PYGZgt{}\PYGZgt{} }\PYG{n}{moleq} \PYG{o}{=} \PYG{n}{optimize}\PYG{p}{(}\PYG{n}{solmf}\PYG{p}{,} \PYG{n}{maxstep}\PYG{o}{=}\PYG{l+m+mi}{100}\PYG{p}{)}
\end{sphinxVerbatim}


\section{Thermochemistry}
\label{\detokenize{examples:thermochemistry}}
\sphinxAtStartPar
We provide a code for numerical Hessian matrix constructed by analytic forces due to the non\sphinxhyphen{}Hermitian Hamiltonian resulted by the self\sphinxhyphen{}energy. Thermochemical properties can be calculated by utilizing \sphinxcode{\sphinxupquote{pyscf.hessian.thermo}} module.
\sphinxcode{\sphinxupquote{Hessian}} offers three\sphinxhyphen{}point (default) and five\sphinxhyphen{}point finite difference method for calculating the Hessian matrix. The following code introduces how to obtain thermochemical properties using \sphinxcode{\sphinxupquote{harmonic\_analysis}} function:

\begin{sphinxVerbatim}[commandchars=\\\{\}]
\PYG{g+gp}{\PYGZgt{}\PYGZgt{}\PYGZgt{} }\PYG{k+kn}{from}\PYG{+w}{ }\PYG{n+nn}{fcdft}\PYG{n+nn}{.}\PYG{n+nn}{hessian} \PYG{n}{numhess} \PYG{k+kn}{import}\PYG{+w}{ }\PYG{o}{*}
\PYG{g+gp}{\PYGZgt{}\PYGZgt{}\PYGZgt{} }\PYG{n}{hessmf} \PYG{o}{=} \PYG{n}{Hessian}\PYG{p}{(}\PYG{n}{solmf}\PYG{p}{)}
\PYG{g+gp}{\PYGZgt{}\PYGZgt{}\PYGZgt{} }\PYG{n}{hess} \PYG{o}{=} \PYG{n}{hessmf}\PYG{o}{.}\PYG{n}{kernel}\PYG{p}{(}\PYG{p}{)}
\PYG{g+gp}{\PYGZgt{}\PYGZgt{}\PYGZgt{} }\PYG{k+kn}{from}\PYG{+w}{ }\PYG{n+nn}{pyscf}\PYG{n+nn}{.}\PYG{n+nn}{hessian}\PYG{n+nn}{.}\PYG{n+nn}{thermo}\PYG{+w}{ }\PYG{k+kn}{import} \PYG{n}{harmonic\PYGZus{}analysis}
\PYG{g+gp}{\PYGZgt{}\PYGZgt{}\PYGZgt{} }\PYG{n}{freq\PYGZus{}info} \PYG{o}{=} \PYG{n}{harmonic\PYGZus{}analysis}\PYG{p}{(}\PYG{n}{moleq}\PYG{p}{,} \PYG{n}{hess}\PYG{p}{)}
\end{sphinxVerbatim}

\sphinxAtStartPar
Once the quantities are obtained, these can be saved into a molden format as implemented in \sphinxcode{\sphinxupquote{fcdft.tools.molden}}:

\begin{sphinxVerbatim}[commandchars=\\\{\}]
\PYG{g+gp}{\PYGZgt{}\PYGZgt{}\PYGZgt{} }\PYG{k+kn}{from}\PYG{+w}{ }\PYG{n+nn}{fcdft}\PYG{n+nn}{.}\PYG{n+nn}{tools}\PYG{n+nn}{.}\PYG{n+nn}{molden}\PYG{+w}{ }\PYG{k+kn}{import} \PYG{n}{dump\PYGZus{}freq}
\PYG{g+gp}{\PYGZgt{}\PYGZgt{}\PYGZgt{} }\PYG{n}{dump\PYGZus{}freq}\PYG{p}{(}\PYG{n}{moleq}\PYG{p}{,} \PYG{n}{freq\PYGZus{}info}\PYG{p}{,} \PYG{l+s+s1}{\PYGZsq{}}\PYG{l+s+s1}{freq.molden}\PYG{l+s+s1}{\PYGZsq{}}\PYG{p}{)}
\end{sphinxVerbatim}

\sphinxstepscope


\chapter{About}
\label{\detokenize{about:about}}\label{\detokenize{about::doc}}
\sphinxAtStartPar
Fractional Charge Density Functional Theory (FC\sphinxhyphen{}DFT) is a theory that reformulates open quantum systems in terms of the canonical ensemble. The prototype of FC\sphinxhyphen{}DFT is linear interpolation FC\sphinxhyphen{}DFT (LI\sphinxhyphen{}FC\sphinxhyphen{}DFT), which enforces the Perdew\sphinxhyphen{}Parr\sphinxhyphen{}Levy\sphinxhyphen{}Balduz (PPLB) condition to DFT calculations through linear interpolation and bypasses the delocalization error.


\section{Developers}
\label{\detokenize{about:developers}}
\sphinxAtStartPar
Jun\sphinxhyphen{}Hyeong Kim, Duke University

\sphinxAtStartPar
Weitao Yang, Duke University


\section{Citation}
\label{\detokenize{about:citation}}
\sphinxAtStartPar
Jun\sphinxhyphen{}Hyeong Kim, Dongju Kim, Weitao Yang, and Mu\sphinxhyphen{}Hyun Baik. Fractional Charge Density Functional Theory and Its Application to the Electro\sphinxhyphen{}inductive Effect. \sphinxstyleemphasis{J. Phys. Chem. Lett.} \sphinxstylestrong{2023}, \sphinxstyleemphasis{14}, 3329\sphinxhyphen{}3334

\sphinxAtStartPar
Jun\sphinxhyphen{}Hyeong Kim and Weitao Yang. Fractional Charge Density Functional Theory Elucidates Electro\sphinxhyphen{}Inductive and Electric Field Effects at Electrochemical Interfaces. \sphinxstyleemphasis{To be submitted}


\section{Bug reports and feature requests}
\label{\detokenize{about:bug-reports-and-feature-requests}}
\sphinxAtStartPar
Please create a posting on \sphinxhref{https://github.com/Yang-Laboratory/FC-DFT/issues}{Issues} tab.


\chapter{Indices and tables}
\label{\detokenize{index:indices-and-tables}}\begin{itemize}
\item {} 
\sphinxAtStartPar
\DUrole{xref}{\DUrole{std}{\DUrole{std-ref}{genindex}}}

\item {} 
\sphinxAtStartPar
\DUrole{xref}{\DUrole{std}{\DUrole{std-ref}{modindex}}}

\item {} 
\sphinxAtStartPar
\DUrole{xref}{\DUrole{std}{\DUrole{std-ref}{search}}}

\end{itemize}



\renewcommand{\indexname}{Index}
\printindex
\end{document}